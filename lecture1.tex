\section*{Семинар 1.}
\subsection*{Линейные программы}


\begin{definition}
	\emph{Линейной программой в общей форме} называется линейная программа вида:
\end{definition}

\[
	\begin{cases}
		x_j \in \R / \R_+ / \R_-\\
		\sum_{j = 1}^{n} a_{i j} x_j \leq = \geq b_i\\
		\sum_{j = 1}^n c_j x_j \to \max / \min
	\end{cases}
	\Leftrightarrow
	\begin{cases}
		x \in \R^n_{?}\\
		A x \leq = \geq b \\
		c^T x \to \max / \min
	\end{cases}
\]

\begin{definition}
	\emph{Линейной программой в стандартной форме} называется линейная программа вида:
\[
	\begin{cases}
		x \in \R^n_+\\
		A x = b\\
		c^T x \to \max
	\end{cases}
\]

$rk A = m$
\end{definition}

Произведем руками переход из общей в стандартную:
\[
	\begin{cases}
		x_1 \in	\R, x_2 \in \R_+, x_3 \in \R_+\\
		2 x_1 - x_3 \leq 5\\
		x_2 - x_3 = 4\\
		x_1 + x_3 \to \min
	\end{cases}
\]

тут пару преобразований, типо $x_1 = y_1 - z_1$

\[
	\begin{cases}
		x_1, z_1, x_2, x_3, t \in \R_+\\
		2 y_1 - 2 z_1 - x_3 + t_0 = 5\\
		x_2 - x_3 = 4\\
		-y_1 + z_1 - x_3 \to \max
	\end{cases}
\]

Новый пример:
\[
	\begin{cases}
		x_1 \in \R, x_2 \in \R_+
		x_1 - x_2 \geq 1\\
		x_2 \leq 2
	\end{cases}
\]

надо понять где выглядит решение

тут картинка $0 \leq x_2$

тут картинка $ \leq 2$

тут картинка $x_1 - x_2 \leq 1$

тут картинка их мержа

То что получили не ограничено

\begin{definition}
	\emph{Полиэдр} -- множество полученное пересечением линейных условий(неравенств)
\end{definition}


\begin{definition}
	\emph{Политоп} -- Ограниченный полиэдр
\end{definition}

Сказать что полиэдр это первые две строки из Стандартной LP

Выпуклый ли Полиэдр?
\begin{enumerate}
	\item Доказываем что полупространство выпукло
	\item Пересечение выпуклых -- выпукло
	\item Значит полиэдр выпукл
\end{enumerate}

Полиэдр бывает точкой(1, 2)

Полиэдр бывает пустым($x_3 >= 3$)

Пусть $P$ -- полиэдр

\[
	\begin{cases}
		x \in P\\
		c^T x \to \max
	\end{cases}
\]

Бывает 3 варианта развития событий:
\begin{enumerate}
	\item Задача несовместна(полиэдр пустой)
	\item Задача ограничена(сколь угодно большой(по модулю) макс ограничен)
	\item Задача не ограничена(мин ограничен)
\end{enumerate}
Тут пример 1): 
\[
	\begin{cases}
		x_1, x_2 \in \R_+\\
		x_1 - x_2 \leq 1
		2 x_1 + x_2 \to \max
	\end{cases}	
\]

Значение может быть сколь угодно большим

Теперь пример 2)
\[
	\begin{cases}
		x_1, x_2 \in \R_+\\
		x_1 - x_2 \leq 1
		2 x_1 + x_2 \to \min
	\end{cases}	
	\Leftrightarrow
	\begin{cases}
		x_1, x_2 \in \R_+\\
		x_1 - x_2 \leq 1
		-2 x_1 - x_2 \to \max
	\end{cases}	
	\Leftrightarrow
	\begin{cases}
		x_1, x_2 \in \R_+\\
		x_1 - x_2 \leq 1
		\left< \left(-2, -1\right) \left(x_1, x_2\right) \right> \to \max
	\end{cases}
\]
Значит нужна точка наиболее удаленная от вектора $\left(-2, -1\right)$

Для политопа любая задача ограничен или несовместна

\section*{Задачи в стандартной форме}
\[
	\begin{cases}
		x \in \R_+^N\\
		A x = b\\
	\end{cases}
\]

Сказ про то что условия неотрицательности не надо пихать в тело а надо наверх

Def $A \in \Mat_{m \times n}$, тогда $A$ называется \emph{тотально унимодулярной} 
если для любого минора матрицы его определитель лежит в $\{-1, 0, 1\}$

Th Если P -- 
\[
	\begin{cases}
		x \in \R_+^N\\
		A x = b\\
		A -- TU\\
		b \in Z^m
	\end{cases}
\]
То все вершины $P$ целочислены

Def $v \in P$ -- \emph{вершина} если $\forall \eps > 0$ выполнено $v + \eps \not \in P$ или $v - \eps \not \in P$.

Th Сильное утверждение про вершины с лекции

тут картиночка прямоугольника матрицы A, вектора x(высокий) = b столбец

Хотим по точке $x$ понять вершина это или нет.
$z(x)$ -- нулевые координаты
$nz(x)$ --  Не нулевые
$A_{nz(x)}$ оставили только столбцы с не нулевыми координаты
$rk A_{nz(x)} = |nz(x)| = m$

To be continued

\section*{Двойственная программа}

LP в общей форме

Картинка матрицы $A$, справа вектор-стобец свободных коэфов $b$
снизу целевой функционал с $n$ переменными

Это прямая(англ аналог?) линейная программа

Сопоставим некоторую другую

Картинка матрицы $A^T$, выходит $m$ перменных стоблец свободных коэфов вектор целевого функционала
Вектор $b$ станет $b^T$ снизу

двойственная меняет $\max$ на $\min$
Это двойственная или dual линейная программа

Табличка соответствия
P | D
Переменные | Условия
Условия | Переменные

Def условие ``естественно'' если оно ``мешает'' оптимизировать

\[
	\begin{cases}
	P: \begin{cases}
		x_1 + x_2 \leq 5\\
	\end{cases}\\
	4x_1 + 5x_2 \to max
	\end{cases}
\]

Для $\max$ есть $\leq$, для $\min$ есть $\geq$
$\R_+$ естественные условия
$\R_-$ противоестественные условия
$\R$ это равенства
Работает в две стороны

\[
	\begin{cases}
		x_1 \in \R, x_2 \in R_{-}\\
		2x_2 + x_2 = 5\\
		x_1 \leq 4
		3 x_1 - x2 \leq -2
		x_1 + x_2 \to \min
	\end{cases}
	\Leftrightarrow
	\begin{cases}
		\left(\begin{matrix}
		 2 & 1\\
		 1 & 0\\
		 3 & -1
		\end{matrix}\right)
	\end{cases}
\]
