\section{Семинар 7}

\subsection{Симплекс метод}

\[
    \begin{cases}
        A \in \R_{m \times n}\\
        A x \leq b\\
        x \in \R^n_+\\
        c^T x \to \max
    \end{cases}    
\]

Паралельно будем показывать пример

\[
    \begin{cases}
        x_1, x_2 \in \R_+\\
        x_1 - x_2 \leq 2\\
        2x_1 - x_2 \leq 6\\
        3x_1 + 2 x_2 \leq 23\\
        2x_1 + x_2 \to \max
    \end{cases}  
\]

Сделаем довольно интресный переход к \emph{канонической форме}

\[
    \begin{cases}
        x_1, x_2, x_3, x_4, x_5 \in \R\\
        x_3 = -x_1 + x_2 + 2\\
        x_4 = -2x_1 + x_2 + 6\\
        x_5 = -3x_1 - 2x_2 + 23\\
        2 x_1 + x_2 \to \max
    \end{cases}    
\]

$x_3, x_4, x_5$ называются базисными, и они не встречаются с другой стороны от равенства.
Другие переменные называются не базисными.

В любой момент времение у нас в Симплекс-методе будет содержать $m$ базисных и $n$ не базисных

\subsubsection{Шаг симплекс алгоритма}

Для канонической формы назовем \emph{базисным решением}, такое решение, что все не базисным переменные равны 0
Для нашей задачи это: $\left(0, 0, 2, 6, 23\right)$.

Текущее преположение: $b \leq 0$. Оно гарантирует следующий факт: начальное базисное решение -- допустимо.

Будем шевелить $x_1$ увеличивая остальные не базисные не трогаем, тогда ограничения на $x_1$:
\[
    \begin{cases}
        x_1 \leq 2\\
        x_1 \leq 3\\
        x_1 \leq \frac{23}{3}
    \end{cases}
\]
Самое жесткое ограничение из-за $x_3$. Перекинем $x_1$ и $x_3$ тогда
\[
    \begin{cases}
        x_1 = -x_3 + x_2 + 2\\
        x_4 = -2x_1 + x_2 + 6\\
        x_5 = -3x_1 - 2x_2 + 23\\
    \end{cases}  \to
    \begin{cases}
        x_1 = -x_3 + x_2 + 2\\
        x_4 = 2x_3 - x_2 + 2\\
        x_5 = 3x_3 - 5x_2 + 17\\
        -2x_3 + 3 x_2 + 4 \to \max\\
    \end{cases}
\]

Новое БР: $\left(2, 0, 0, 2, 17\right)$
За счет того что мы всегда берем самое жесткое ограничение то БР тоже допустимо.
Ну и текущий оптимум это 4.

Поехали делать это для $x_2$:
\[
    \begin{cases}
        x_2 \leq \infty\\
        x_2 \leq 2\\
        x_2 \leq \frac{17}{5}\\
    \end{cases}
\]

Значит свапаем $x_2, x_4$($x_2$ -- вводимое и $x_4$ выводимое)

\[
    \begin{cases}
        x_1 = -x_3 + x_2 + 2\\
        x_2 = 2x_3 - x_4 + 2\\
        x_5 = 3x_3 - 5x_2 + 17\\
        -2x_3 + 3 x_2 + 4 \to \max\\
    \end{cases} \to
    \begin{cases}
        x_1 = x_3 - x_4 + 4\\
        x_2 = 2x_3 - x_4 + 2\\
        x_5 = -7x_3 + 5x_4 + 7\\
        4x_3 - 3 x_4 + 10 \to \max\\
    \end{cases}
\]

БР -- $\left(4, 2, 0, 0, 7\right)$
\[\begin{cases}
    x_3 \leq \infty\\
    x_3 \leq \infty\\
    x_3 \leq 1\\
\end{cases}\]

Меняем $x_3, x_5$:

\[
    \begin{cases}
        x_1 = x_3 - x_4 + 4\\
        x_2 = 2x_3 - x_4 + 2\\
        x_5 = -7x_3 + 5x_4 + 7\\
        4x_3 - 3 x_4 + 10 \to \max\\
    \end{cases}\to
    \begin{cases}
        x_3 = -\frac17 x_5 - \frac27 x_4 + 5\\
        x_1 = -\frac17 x_5 + \frac57 x_4 + 1\\
        x_2 = -\frac27 x_5 + \frac37 x_4 + 4\\
        -\frac47 x_5 -\frac17 x_4 + 14 \to \max\\
    \end{cases}
\]
БР: $(5, 4, 1, 0, 0)$ и очевидно, что $-\frac47 x_5 -\frac17 x_4 + 14 \leq 14$, значит БР оптимально.

\subsubsection{Рубрика утверждения без доказательства}
Если у меня есть ограничение которое можно увеличивать хоть чуть-чуть то увеличиваем

Иначе это вырожденное решение, делаем следующее: $(i, j)$ -- пара свопов допустимая, минимальная лексикографически. 
Правило Бленда, в общих чертах -- рассмотрим множество базисных переменнных, это сочетания, они вжух там увеличиваются

Не ну ежу понятно что это тетта, даже Глеб бы справился.

\subsubsection{Kek trick}

\[
    \begin{cases}
        x_{n + 1} = (a_{1, 1}x_1 + \dots a_{1, n} x_n) + b_1\\
        x_{n + 2} = (a_{2, 1}x_1 + \dots a_{2, n} x_n) + b_2\\
        \dots\\
        x_{n + m} = (a_{m, 1}x_1 + \dots a_{m, n} x_n) + b_m\\
        c_1 x_1 + \dots + c_n x_n \to \max
    \end{cases}\to
    \begin{cases}
        x_{n + 1} = (a_{1, 1}x_1 + \dots a_{1, n} x_n) + \lambda + b_1\\
        x_{n + 2} = (a_{2, 1}x_1 + \dots a_{2, n} x_n) + \lambda + b_2\\
        \dots\\
        x_{n + m} = (a_{m, 1}x_1 + \dots a_{m, n} x_n) + \lambda + b_m\\
        -\lambda \to \max
    \end{cases}
\]

Новая система называются \emph{вспомогательной или присоединенной системой}
Она всегда совместна, и $\lambda = 0 \Leftrightarrow$ полиэдр совместный.

Дальше берем и свапаем $\lambda$ в минимальной $b$шке. 
Теперь у нас из каждой $b$шки вычтеться минимальная значит они все не меньше 0. Можем крутить симплекс метод.

Теперь если оптимум равен 0 есть два варианта:
1) $\lambda$ -- не базисная в конце. Ну тогда берем текущую систему и вычеркиваем $\lambda$.

2) $\lambda$ -- базисная в конце. Тогда сделаем одну любую в строчке с $\lambda$. 
Очевидно там кто-то есть и свободный коэфициент это 0 из симплекс метода.
