\section{Семинар 2}
\subsection*{Продолжаем про TU}
TH1:
Пусть есть полиэдр
\[
	P = \begin{cases}
		x \in \R^+\\
		A x = b\\
		A \in Mat_{m \times n}
	\end{cases}
\]

$A -- TU$ и $b \in \Z^+$, тогда полиэдр целый(любая вершина целая)

Lm1: $x$ -- вершина полиэдра $\Leftrightarrow$ $A_{nz(x)}$ -- Линейно Независимые столбцы (упражнение)
$\Leftrightarrow$ $\exists$ базисное допустимое решение $(x, B)$, и:
\[
	\begin{cases}
		nz(x) \subseteq B	\\
		|B| = m\\
		det A_B \ne 0
	\end{cases}
\]


Проебал док-во Th1


Пример:
Есть двудольный граф
$G = (V, E)$, $V = L \bigsqcup R$
def Политоп парасочетаний
Мы будем работать с векторами у которых
1. Мощность $E$ у каждого вектора
2. $x_e \geq 0$
3. $\sum_{e \in \sigma(v)} x_e \leq 1$
Буде обознать $M(G)$ этот политоп


def Далее есть политоп $PM(G)$ -- политоп совершенных паросочетаний про него
1. Мощность E
2. $x_e \geq 0$
3. $\sum_{e \in \sigma(v)} x_e = 1$
Чтобы он был(ну не пустым) должно быть:
1. $|L| = |R|$ -- посчитали сумму всех ребер, с одной стороны это размер левой доли, с другой стороны с правой

Окей пусть $x \in M(G)$, -- целая точка. Но тогда $x \to$ некоторое множество ребер. Как битовую маску к примеру.
Пусть множество ребер которых получилось $A = \left\{e | x_e = 1\right\}$
С другой стороны берем парсоч и переводим его в точку политопа как битовую маску.

Тогда заметим что целые точки в $PM(G)$ соответствуют совершенными парсочетаниям

Th $PM(G)$ -- целый
Lm матрица $PM(G) -- TU$. 
Заметим что $PM(G)$ -- политоп заданный в стандартной форме.
Итак в столбце будет ровно две единички. То есть это матрица инцидентности
Докажем по индукции. Если в миноре есть столбец без единиц, определитель 0.
Если есть столбец с одной единицей то разложим по ней
Если нет то в нём есть цикл, значит есть линейно заивисимые столбцы, а значит определитель 0
Ну мы закончили

Th $M(G)$
Ну сначала переведем в стандартную форму:
\[
	\widetilde{M}(G) = \begin{cases}
		\forall e: x_e \leq 0,\\
		s_v \leq 0,\\
		\sum_{e \in \sigma(v)} x_e + s_v = 1\\
	\end{cases}
\]
 Очевидно что есть биекция $M(G) \leftrightarrow \widetilde{M}(G)$, докажем что вершина уйдет в вершину.

 Докажем что не вершина соответствует не вершине.
 Итого $x$ не вершина, тогда $\exists [x - \eps, x + \eps] \subseteq M(G)$
 Итого $y$ не вершина, тогда $\exists [(x - \eps, s - \delta), (x + \eps, s + \delta)] \subseteq \widetilde{M}(G)$
 надо пруфануть

 Итак получается таже самая матрица и справа единичная, раскладываем сначала по ним дальше очев предидущая.

 \subsection{Профит с целых полиэдров}
 Итак пусть у нас есть задача поиска максимального паросочетания
 \[
 \begin{cases}
 	x_e \in \{0, 1\} \Leftrightarrow x_e \in \Z, x_e \leq 0, x_e \leq 1\\
 	\sum_{x_e} \leq 1\\
 	e \in \sigma(v)\\
 	\sum x_e \to \max
 \end{cases}
 \]

 Это задача целочисленного ЛП. Давайте возьмем и сделаем другую задачу
 \[
 \begin{cases}
 	x_e \in \R^+ \\
 	\sum_{x_e} \leq 1\\
 	e \in \sigma(v)\\
 	\sum x_e \to \max
 \end{cases}
 \]
 Мы достигаем того-же максимума и можем это сделать это на вершине.
 Мы сделали релаксацию от целочисленной к линейной. И это способ валиден только если полиэдр целый.

 Упр. 
 Проверить наличие $K_{n}$ в графе


Двойственная к $M(G)$ это политоп вершинного покрытия

Убедиться что мы умеем искать минимальное вершинное покрытие по макс парасочетанию

Итак давай-те зашарашим условия доп нежесткости
пусть $x, y$ -- оптимальны, тогда
\[
	\begin{cases}
		x_e (y_u + y_v - 1) = 0\\
		y_v (\sum_{e\in\sigma(v)}x_e - 1) = 0
	\end{cases}
\]


