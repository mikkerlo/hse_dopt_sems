\section*{Задача про петухов на графе}

\subsection*{Что-то простое для решения, 2-приблежение через MST}

\subsection*{Christofidis}

Работает если $C$ это метрика

Возьмем MST(G), и обозначим за $A = \{V | (2 | deg)\}$

Возьмем $M$ парасочетание в $A$, нужно $C(M) \to \min$

Предположим что существует полиноминальный алгоритм поиска парсоча 
в полном графе.

Итак мы добавили эти ребра к МСТ, теперь есть эйлеров цикл.

Далее сделаем гамильтонов из эйлерова, за счет свойств метрики и
правил построения, его стоймость не выше чем стоимость элейлерова:
\[
    C(P) \leq C(ET) = C(T) + C(M) \leq OPT + C(M) \leq \frac32 OPT    
\]

Осталось доказать что $C(M) \leq \frac12 OPT$, то что $C(T) \leq OPT$
доказывалось в предидущем пункте.

Ну тут просто, нарисуем вершинки, и соединим подряд как нибудь,
тогда $C(M_1) + C(M_2) \leq OPT$. По Дирихле какой-то меньше пополам.

\subsection*{Преобразование K-OPT}

Это всё на плоскости

\subsubsection*{Преобразование 2-OPT}

Пока существует способ поменять пару ребер на другую пару ребер, так
что стоимость уменьшается -- применяю это.

На практике сходится за квадрат итераций.

Далее количество итераци оценивается как $O(n^3)$.

\textit{Кулл стори Боб}: Не ну на практике эта штука это 
$\log n / 2$  приближение

\textit{Кулл стори Боб2}: Если вначале избавиться от всех пересечений,
то останется довольно мало итераций.

\textit{Фу геома}: существует алгоритм который 
квазилинеен и находит все пересечения

\subsubsection*{Преобразование 3-OPT}

Теперь возьмем три ребра, и 
соединить оставшиеся вершины(можно через 1 в порядке их записи)

\subsubsection*{Lin-Kernigan}
Lin-Kernigan -- зачем себе заранее усложнять жизнь и заранее прибивать
контстанту OPT

Щас будет эпик эвристика, ну что го пронумеруем вершинки:
